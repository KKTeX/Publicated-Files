\documentclass[a4paper,12pt]{article}

\usepackage[dvipsnames, svgnames, x11names]{xcolor}
\usepackage{tikz}
\usetikzlibrary{shapes, positioning, shadows, shadows.blur, patterns, decorations.text, 
                decorations.pathmorphing, arrows.meta, calc, snakes, intersections}
\usepackage{xparse, calc, ifthen}
\usepackage{fp}
\usepackage{zref-savepos}
\usepackage{geometry}
\usepackage{hyperref}
\usepackage{lipsum}
\usepackage[most]{tcolorbox}
\usepackage{keisennote}
\usepackage{listings}
\lstset{
    basicstyle=\ttfamily\small,
    keywordstyle=\color{blue},
    commentstyle=\color{gray},
    stringstyle=\color{red},
    breaklines=true,
    breakatwhitespace=false,  
    columns=flexible           
}

\usepackage{hyperref} 
\hypersetup{
  luatex, pdfencoding=auto, 
  colorlinks=true,
  linkcolor=black,     
  citecolor=black,     
  urlcolor=DeepSkyBlue3,      
  pdfborder={0 0 0}, 
}



\title{\texttt{keisennote} Package Documentation}
\author{KKTeX}
\date{Version 1.0.3 (2025/09/17)}
\begin{document}
\begin{titlepage}
  \maketitle
\end{titlepage}
\newpage
\tableofcontents
\newpage

\section{Acknowledgements / Credit}

This package is based on the code from 
\href{https://qiita.com/VoD/items/6849e63b978050218d2f}{VoD's Qiita article}, 
with some improvements. The original author has kindly granted permission 
to release this as a LaTeX package.

\section{Installation}

Place \texttt{keisennote.sty} in a directory where LaTeX can find it, e.g., your local \texttt{texmf} tree or alongside your document.

Dependencies:
\begin{itemize}
    \item \texttt{xcolor} 
    \item \texttt{tikz}
    \item \texttt{xparse}, \texttt{calc}, \texttt{ifthen}
    \item \texttt{fp}
    \item \texttt{zref-savepos} 
    \item \texttt{luatex85}, \texttt{url}, \texttt{expl3}, \texttt{xkeyval}
\end{itemize}


Load the package:

\begin{verbatim}
\usepackage{keisennote}
\end{verbatim}

\section{Commands}

\subsection{\texttt{\textbackslash notefill}}
\begin{verbatim}
\notefill[<scale>][<color>]
\end{verbatim}

Fills the current vertical space with ruled notebook lines and dots.

\begin{itemize}
    \item \texttt{<scale>} (optional, default: 0.5pt): size of triangular end markers.
    \item \texttt{<color>} (optional, default: white!70!black): color of lines and dots.
\end{itemize}

\textbf{Example:}
\begin{verbatim}
\notefill[0.6pt][Gray]
\end{verbatim}

\subsection{\texttt{\textbackslash note}}
\begin{verbatim}
\note{<lines>}[<scale>][<color>]
\end{verbatim}

Typesets a short ruled block with a specified number of lines.

\begin{itemize}
    \item \texttt{<lines>} (mandatory, integer $\ge$ 2): number of ruled lines.
    \item \texttt{<scale>} (optional, default: 0.5pt): size of triangular markers.
    \item \texttt{<color>} (optional, default: white!70!black): color of lines and dots.
\end{itemize}

\textbf{Example:}
\begin{verbatim}
\note{5}[0.4pt][NavyBlue]
\end{verbatim}

This produces the following output.\bigskip

\note{5}[0.4pt][NavyBlue]

\bigskip Inserting \verb|\bigskip| before (and after) using the \verb|\note| command can sometimes improve the appearance.

\subsection{\texttt{\textbackslash masumefill}}
\begin{verbatim}
\masume[<scale>][<color>]
\end{verbatim}

Fills the current vertical space with grids and dots.

\begin{itemize}
    \item \texttt{<scale>} (optional, default: 0.5pt): size of triangular end markers.
    \item \texttt{<color>} (optional, default: white!70!black): color of lines and dots.
\end{itemize}

\textbf{Example:}
\begin{verbatim}
\notefill[0.6pt][Gray]
\end{verbatim}

\subsection{\texttt{\textbackslash masume}}
\begin{verbatim}
\masume{<lines>}[<scale>][<color>]
\end{verbatim}

Typesets a short grid block with a specified number of lines.

\begin{itemize}
    \item \texttt{<lines>} (mandatory, integer $\ge$ 2): number of ruled lines.
    \item \texttt{<scale>} (optional, default: 0.5pt): size of triangular markers.
    \item \texttt{<color>} (optional, default: white!70!black): color of lines and dots.
\end{itemize}

\textbf{Example:}
\begin{verbatim}
\masume{5}[0.4pt][NavyBlue]
\end{verbatim}

This produces the following output.\bigskip

\masume{5}[0.4pt][NavyBlue]

\bigskip Inserting \verb|\bigskip| before (and after) using the \verb|\masume| command can sometimes improve the appearance.

\section{Package Parameters}

These dimensions can be adjusted:

\begin{itemize}
    \item \texttt{\textbackslash noteLineWidth}: thickness of ruled lines (default: 0.5pt)
    \item \texttt{\textbackslash dotsRadius}: radius of intersection dots (default: 0.8pt)
    \item \texttt{\textbackslash noteLineDistance}: vertical distance between lines (default: 6mm)
\end{itemize}

Example:
\begin{verbatim}
\setlength{\noteLineDistance}{7truemm} % A-kei spacing
\end{verbatim}

\section{Examples}

\subsection{Short Note Block}
\begin{verbatim}
\note{4}
\end{verbatim}
\note{4}

\subsection{Full Page Fill}
\begin{verbatim}
\notefill
\end{verbatim}
\notefill

\newpage\section{Implementation Notes}

\begin{itemize}
    \item Notebook lines are drawn using TikZ, with dots placed at equal horizontal intervals.
    \item The number of dots per line is automatically calculated using the \texttt{fp} package.
    \item Triangular markers are added at the top and bottom of each ruled block.
    \item \texttt{\textbackslash notefill} measures available vertical space using \texttt{zref-savepos}.
\end{itemize}

\section{License}

Released under the \href{https://www.latex-project.org/lppl/}{LaTeX Project Public License (LPPL) 1.3c}.

\section{Version History}

\begin{itemize}
    \item \textbf{v1.0.0 (2025/09/13)} --- Initial public release.
    \item \textbf{v1.0.3 (2025/09/13)} --- KKTeX added \texttt{\textbackslash masume} and \texttt{\textbackslash masumefill}.
\end{itemize}

\section{Source Code}
  \begin{lstlisting}
    \ProvidesPackage{keisennote}[2025/09/17, v1.0.3]

    \RequirePackage[dvipsnames, svgnames, x11names]{xcolor}
    \RequirePackage{luatex85, zref, zref-savepos, fp, url, expl3, xkeyval}
    \RequirePackage{tikz}\RequirePackage{graphicx}
    \usetikzlibrary{shapes, positioning, shadows, shadows.blur, patterns, decorations.text, decorations.pathmorphing, arrows.meta, calc, snakes, intersections}
    \RequirePackage{xparse, calc, ifthen}

    \newdimen\VoD@mag
    \VoD@mag=.5pt 
    \newdimen\noteLineWidth
    \newdimen\dotsRadius
    \newdimen\noteLineDistance
    \noteLineWidth.5truept\relax%    <- 罫線の太さ
    \dotsRadius.8truept\relax%       <- ドットの半径
    \noteLineDistance=6truemm\relax% <- 罫線間隔(A罫 : 7truemm,B罫 : 6truemm) 

    %%%必要な内部レジスタの用意
    \newdimen\VDNT@currentXPos
    \newdimen\VDNT@currentYPos
    \newdimen\VDNT@Xinterval
    \newdimen\VDNT@Yinterval
    \newdimen\VDNT@notegoal


    %%% \notefillで用いる座標管理用カウンタの準備
    \def\VDNT@pkgname{vodnote}
    \global\newcount\VDNT@uniqe


    %%% \notefill の定義
    \NewDocumentCommand{\notefill}{ O{.5pt} O{white!70!black} }{\par\bgroup
      \VoD@mag=#1
      \parindent\z@
      %%罫線間隔の算出
      \@tempcnta\linewidth
      \@tempcntb\noteLineDistance
      \FPeval\VDNT@dotsNum{round(round(((\the)\@tempcnta/(\the)\@tempcntb)/2:0)*2:0)}%
      \VDNT@Xinterval\dimexpr(\linewidth)/\VDNT@dotsNum\relax
      \VDNT@Yinterval\VDNT@Xinterval
      %%上端の座標取得
      \zsaveposy{\VDNT@pkgname.\the\VDNT@uniqe.TopPos}%
      %%下端の座標取得
      \leavevmode\vfill\leavevmode
      \zsaveposy{\VDNT@pkgname.\the\VDNT@uniqe.BottomPos}%
      %%ノート罫線描画幅の決定
      \VDNT@notegoal=\dimexpr
        \zposy{\VDNT@pkgname.\the\VDNT@uniqe.TopPos}sp
        -\zposy{\VDNT@pkgname.\the\VDNT@uniqe.BottomPos}sp
      \relax
      %%ノート罫線描画
      \noindent\smash{%
        \begin{tikzpicture}[xscale=0.996]
          \VDNT@currentYPos\z@
          \fill[#2] (\VDNT@Xinterval*\VDNT@dotsNum/2,\VDNT@currentYPos+\VoD@mag*4pt) -- ++(\VoD@mag*3pt,-\VoD@mag*4pt) -- ++(-\VoD@mag*6pt,0) -- cycle;
          \@whiledim\VDNT@currentYPos<\VDNT@notegoal\do{
            \VDNT@currentXPos\z@
            \draw[#2,line width=\noteLineWidth] (0,\VDNT@currentYPos) -- (\linewidth,\VDNT@currentYPos);
            \foreach \k in{0,1,...,\VDNT@dotsNum}{%
              \VDNT@currentXPos=\dimexpr\VDNT@Xinterval*\k\relax
              \fill[#2] (\VDNT@currentXPos,\VDNT@currentYPos) circle [radius=\dotsRadius];
            }
            \advance\VDNT@currentYPos\VDNT@Yinterval\relax
          }
          \fill[#2] (\VDNT@Xinterval*\VDNT@dotsNum/2,\VDNT@currentYPos-\VDNT@Yinterval-\VoD@mag*4pt) -- ++(\VoD@mag*3pt,\VoD@mag*4pt) -- ++(-\VoD@mag*6pt,0) -- cycle;
        \end{tikzpicture}%
      }%
      \egroup
      %%座標管理用カウンタのインクリメント
      \global\advance\VDNT@uniqe\@ne
      \par
    }


    %%% \note の定義(2以上の整数を引数に)
    \NewDocumentCommand{\note}{ m O{.5pt} O{white!70!black} }{\par\bgroup
      %%三角形の大きさ
      \VoD@mag=#2
      %%罫線間隔の算出
      \@tempcnta\linewidth
      \@tempcntb\noteLineDistance
      \FPeval\VDNT@dotsNum{round(round(((\the)\@tempcnta/(\the)\@tempcntb)/2:0)*2:0)}%
      \VDNT@Xinterval\dimexpr\linewidth/\VDNT@dotsNum\relax
      \VDNT@Yinterval\VDNT@Xinterval
      %%ノート罫線描画
      \noindent
        \begin{tikzpicture}[xscale=0.996]
          \VDNT@currentYPos\z@
          \fill[#3] (\VDNT@Xinterval*\VDNT@dotsNum/2,\VDNT@currentYPos+\VDNT@Yinterval+\VoD@mag*4pt) -- ++(\VoD@mag*3pt,-\VoD@mag*4pt) -- ++(-\VoD@mag*6pt,0) -- cycle; %上の三角形
          \foreach \i in{1,2,...,#1}{ 
            \VDNT@currentXPos\z@
            \global\VDNT@currentYPos=\dimexpr\VDNT@Yinterval*\i\relax
            \draw[#3,line width=\noteLineWidth] (0,\VDNT@currentYPos) -- (\linewidth,\VDNT@currentYPos);
            \foreach \k in{0,1,...,\VDNT@dotsNum}{
              \VDNT@currentXPos=\dimexpr\VDNT@Xinterval*\k\relax
              \fill[#3] (\VDNT@currentXPos,\VDNT@currentYPos) circle [radius=\dotsRadius];
            }
          }
          \fill[#3] (\VDNT@Xinterval*\VDNT@dotsNum/2,\VDNT@currentYPos-\VoD@mag*4pt) -- ++(\VoD@mag*3pt,\VoD@mag*4pt) -- ++(-\VoD@mag*6pt,0) -- cycle; %下の三角形
        \end{tikzpicture}%
      \egroup
      \par
    }


    \NewDocumentCommand{\masumefill}{ O{.5pt} O{white!70!black} }{\par\bgroup
      \VoD@mag=#1
      \parindent\z@
      %%罫線間隔の算出
      \@tempcnta\linewidth
      \@tempcntb\noteLineDistance
      \FPeval\VDNT@dotsNum{round(round(((\the)\@tempcnta/(\the)\@tempcntb)/2:0)*2:0)}%
      \VDNT@Xinterval\dimexpr(\linewidth)/\VDNT@dotsNum\relax
      \VDNT@Yinterval\VDNT@Xinterval
      %%上端の座標取得
      \zsaveposy{\VDNT@pkgname.\the\VDNT@uniqe.TopPos}%
      %%下端の座標取得
      \leavevmode\vfill\leavevmode
      \zsaveposy{\VDNT@pkgname.\the\VDNT@uniqe.BottomPos}%
      %%ノート罫線描画幅の決定
      \VDNT@notegoal=\dimexpr
        \zposy{\VDNT@pkgname.\the\VDNT@uniqe.TopPos}sp
        -\zposy{\VDNT@pkgname.\the\VDNT@uniqe.BottomPos}sp
      \relax
      %%ノート罫線描画
      \noindent\smash{%
        \begin{tikzpicture}[xscale=0.996]
          \VDNT@currentYPos\z@
          \fill[#2] (\VDNT@Xinterval*\VDNT@dotsNum/2,\VDNT@currentYPos+\VoD@mag*4pt) -- ++(\VoD@mag*3pt,-\VoD@mag*4pt) -- ++(-\VoD@mag*6pt,0) -- cycle;
          \@whiledim\VDNT@currentYPos<\VDNT@notegoal\do{
            \VDNT@currentXPos\z@
            \draw[#2,line width=\noteLineWidth] (0,\VDNT@currentYPos) -- (\linewidth,\VDNT@currentYPos);
            \foreach \k in{0,1,...,\VDNT@dotsNum}{%
              \VDNT@currentXPos=\dimexpr\VDNT@Xinterval*\k\relax
              \draw[#2,line width=\noteLineWidth]
              (\VDNT@currentXPos,0) -- (\VDNT@currentXPos,\VDNT@notegoal-.5\VDNT@Yinterval);
              \fill[#2] (\VDNT@currentXPos,\VDNT@currentYPos) circle [radius=\dotsRadius];
            }
            \advance\VDNT@currentYPos\VDNT@Yinterval\relax
          }
          \fill[#2] (\VDNT@Xinterval*\VDNT@dotsNum/2,\VDNT@currentYPos-\VDNT@Yinterval-\VoD@mag*4pt) -- ++(\VoD@mag*3pt,\VoD@mag*4pt) -- ++(-\VoD@mag*6pt,0) -- cycle;
        \end{tikzpicture}%
      }%
      \egroup
      %%座標管理用カウンタのインクリメント
      \global\advance\VDNT@uniqe\@ne
      \par
    }


    \NewDocumentCommand{\masume}{ m O{.5pt} O{white!70!black} }{\par\bgroup
      %%三角形の大きさ
      \VoD@mag=#2
      %%罫線間隔の算出
      \@tempcnta\linewidth
      \@tempcntb\noteLineDistance
      \FPeval\VDNT@dotsNum{round(round(((\the)\@tempcnta/(\the)\@tempcntb)/2:0)*2:0)}%
      \VDNT@Xinterval\dimexpr\linewidth/\VDNT@dotsNum\relax
      \VDNT@Yinterval\VDNT@Xinterval
      %%ノート罫線描画
      \noindent
        \begin{tikzpicture}[xscale=0.996]
          \VDNT@currentYPos\z@
          \fill[#3] (\VDNT@Xinterval*\VDNT@dotsNum/2,\VDNT@currentYPos+\VDNT@Yinterval+\VoD@mag*4pt) -- ++(\VoD@mag*3pt,-\VoD@mag*4pt) -- ++(-\VoD@mag*6pt,0) -- cycle; %上の三角形
          \foreach \i in{1,2,...,#1}{ 
            \VDNT@currentXPos\z@
            \global\VDNT@currentYPos=\dimexpr\VDNT@Yinterval*\i\relax
            \draw[#3,line width=\noteLineWidth] (0,\VDNT@currentYPos) -- (\linewidth,\VDNT@currentYPos);
            \foreach \k in{0,1,...,\VDNT@dotsNum}{
              \VDNT@currentXPos=\dimexpr\VDNT@Xinterval*\k\relax
              \draw[#3,line width=\noteLineWidth] (\VDNT@currentXPos,\VDNT@Yinterval) -- (\VDNT@currentXPos,\VDNT@Yinterval*#1);
              \fill[#3] (\VDNT@currentXPos,\VDNT@currentYPos) circle [radius=\dotsRadius];
            }
          }
          \fill[#3] (\VDNT@Xinterval*\VDNT@dotsNum/2,\VDNT@currentYPos-\VoD@mag*4pt) -- ++(\VoD@mag*3pt,\VoD@mag*4pt) -- ++(-\VoD@mag*6pt,0) -- cycle; %下の三角形
        \end{tikzpicture}%
      \egroup
      \par
    }

    \endinput
  \end{lstlisting}
\end{document}
