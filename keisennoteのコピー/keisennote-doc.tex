\documentclass[a4paper,12pt]{article}

\usepackage{booktabs,tabularray}

\usepackage{hyperref}
\usepackage{keisennote}
\usepackage{listings}
\lstset{
    basicstyle=\ttfamily\small,
    keywordstyle=\color{blue},
    commentstyle=\color{gray},
    stringstyle=\color{red},
    breaklines=true,
    breakatwhitespace=false,  
    columns=flexible           
}

\usepackage{hyperref} 
\hypersetup{
  luatex, pdfencoding=auto, 
  colorlinks=true,
  linkcolor=black,     
  citecolor=black,     
  urlcolor=DeepSkyBlue3,      
  pdfborder={0 0 0}, 
}



\title{\texttt{keisennote} Package Documentation}
\author{Kosei Kawaguchi a.k.a. KKTeX}
\date{Version 1.0.4a (2025/10/05)}
\begin{document}
\begin{titlepage}
  \maketitle
\end{titlepage}
\newpage
\tableofcontents
\newpage

\section{Acknowledgements / Credit}

This package is based on the code from 
\href{https://qiita.com/VoD/items/6849e63b978050218d2f}{VoD's Qiita article}, 
with some improvements. The original author has kindly granted permission 
to release this as a LaTeX package.

\section{Installation}

Place \texttt{keisennote.sty} in a directory where LaTeX can find it, e.g., your local \texttt{texmf} tree or alongside your document.

Dependencies:
\begin{itemize}
    \item \texttt{xcolor} 
    \item \texttt{tikz}
    \item \texttt{zref}, \texttt{zref-savepos}, \texttt{fp}
    \item \texttt{kvoptions}
\end{itemize}


Load the package:

\begin{verbatim}
\usepackage{keisennote}
\end{verbatim}

\section{Package options}

This package accepts key--value style options at load time.  
The option handling is powered by \texttt{kvoptions}, with
\texttt{family=kn} and \texttt{prefix=kn@}.  
All options are declared as string options (accepting any \TeX\ length expression)
and are applied during \verb|\ProcessKeyvalOptions*|.  
The available options and their default values are listed below.

\bigskip
\begin{center}
  \begin{tabular}{l l p{.6\linewidth}}
    \toprule
    Option name & Default value & Description \\
    \midrule
    \texttt{linewidth} & \texttt{.5truept}
      & Width of the line used for note drawing.  
        The value is assigned internally to \verb|\noteLineWidth|.  
        Any \TeX\ length (e.g.\ \texttt{1pt}, \texttt{0.6truept}, \texttt{0.2mm}) is accepted. \\
    \texttt{radius} & \texttt{.8truept}
      & Radius of each dot.  
        Internally stored in \verb|\dotsRadius|. \\
    \texttt{distance} & \texttt{6truemm}
      & Spacing between adjacent dots.  
        Internally stored in \verb|\noteLineDistance|. \\
    \texttt{triangle} & \texttt{.5pt}
      & Size of triangular markers.  
        Internally stored in \verb|\VoD@mag|. \\
    \bottomrule
  \end{tabular}
\end{center}
\bigskip

\paragraph{Internal behaviour}
\begin{itemize}
  \item Each option is first stored as a string macro (e.g.\ \verb|\kn@linewidth|), as imposed by \verb|\DeclareStringOption|.
        The package then assigns it to a \verb|\dimen| register, for example:
\begin{lstlisting}[language=TeX]
\noteLineWidth=\kn@linewidth\relax
\dotsRadius=\kn@radius\relax
\noteLineDistance=\kn@distance\relax
\VoD@mag=\kn@triangle\relax
\end{lstlisting}
        This conversion ensures that user-supplied expressions such as
        \texttt{1truept} or \texttt{0.5mm} are properly interpreted as lengths.
  \item If the package does not perform this assignment automatically,
        users may do so manually; however, in normal usage this is handled internally.
\end{itemize}

\paragraph{Examples}
\begin{itemize}
  \item Specify options at package load:
\begin{lstlisting}[language=TeX]
\usepackage[linewidth=1truept, radius=.6truept, distance=8truemm]{keisennote}
\end{lstlisting}
  \item Modify options afterwards using \verb|\setkeys|:
\begin{lstlisting}[language=TeX]
\setkeys{kn}{linewidth=0.8pt, distance=5mm}
\noteLineWidth=\kn@linewidth\relax  % reassign to internal registers if needed
\end{lstlisting}
\end{itemize}

\paragraph{Remarks}
\begin{itemize}
  \item Absolute units such as \texttt{truept} / \texttt{truemm} are used as defaults
        to avoid driver-dependent scaling.
  \item No range checks are performed on the option values.
        Excessively small or negative values may lead to undesirable results.
        If required, minimum-value guards can be implemented via \verb|\ifdim|.
\end{itemize}

\section{Commands}

\subsection{\texttt{\textbackslash notefill}}
\begin{verbatim}
\notefill[<color>]
\end{verbatim}

Fills the current vertical space with ruled notebook lines and dots.

\textbf{Example:}
\begin{verbatim}
\notefill[green]
\end{verbatim}

\subsection{\texttt{\textbackslash note}}
\begin{verbatim}
\note{<lines>}[<color>]
\end{verbatim}

Typesets a short ruled block with a specified number of lines.

\begin{itemize}
    \item \texttt{<lines>} (mandatory, integer $\ge$ 2): number of ruled lines.
    \item \texttt{<color>} (optional, default: white!70!black): color of lines and dots.
\end{itemize}

\textbf{Example:}
\begin{verbatim}
\note{5}[NavyBlue]
\end{verbatim}

This produces the following output.\bigskip

\note{5}[NavyBlue]

\bigskip Inserting \verb|\bigskip| before (and after) using the \verb|\note| command can sometimes improve the appearance.

\subsection{\texttt{\textbackslash masumefill}}
\begin{verbatim}
\masume[<color>]
\end{verbatim}

Fills the current vertical space with grids and dots.

\begin{itemize}
  \item \texttt{<color>} (optional, default: white!70!black): color of lines and dots.
\end{itemize}

\textbf{Example:}
\begin{verbatim}
\notefill[Gray]
\end{verbatim}

\subsection{\texttt{\textbackslash masume}}
\begin{verbatim}
\masume{<lines>}[<color>]
\end{verbatim}

Typesets a short grid block with a specified number of lines.

\begin{itemize}
  \item \texttt{<lines>} (mandatory, integer $\ge$ 2): number of ruled lines.
  \item \texttt{<color>} (optional, default: white!70!black): color of lines and dots.
\end{itemize}

\textbf{Example:}
\begin{verbatim}
\masume{5}[NavyBlue]
\end{verbatim}

This produces the following output.\bigskip

\masume{5}[NavyBlue]

\bigskip Inserting \verb|\bigskip| before (and after) using the \verb|\masume| command can sometimes improve the appearance.

\section{Package Parameters}

These dimensions can be adjusted:

\begin{description}
  \item[\texttt{\textbackslash SetNoteLineWidth}] You can set the width of note lines : \verb|\SetNoteLineWidth[2mm]|
  \item[\texttt{\textbackslash SetNoteDotRadius}] You can set the radius of dots. : \verb|\SetNoteDotRadius[1pt]|
  \item[\texttt{\textbackslash SetNoteLineDistance}] You can set the distance between each lines. \\: \verb|\SetNoteLineDistance[7mm]|
  \item[\texttt{\textbackslash SetNoteTriangleSize}] You can set the size of triangles. : \verb|\SetNoteTriangleSiz[1pt]|
\end{description}

If no argument is given, the parameter is reset to its default value.

\section{Examples}

\subsection{Short Note Block}
\begin{verbatim}
\note{4}
\end{verbatim}
\note{4}
\newpage
\subsection{Full Page Fill}
\begin{verbatim}
\notefill
\end{verbatim}
\notefill\newpage

\section{License}

Released under the \href{https://www.latex-project.org/lppl/}{LaTeX Project Public License (LPPL) 1.3c}.

\section{Version History}

\begin{itemize}
    \item \textbf{v1.0.0 (2025/09/13)} --- Initial public release.
    \item \textbf{v1.0.3 (2025/09/13)} --- KKTeX added \texttt{\textbackslash masume} and \texttt{\textbackslash masumefill}.
    \item \textbf{v1.0.4 (2025/10/4)} --- KKTeX fixed the problem in \verb|\masumefill| and added some package options and setting commands.
    \item \textbf{v1.0.4a (2025/10/5)} --- Added some descriptions about new package options.
\end{itemize}

\section{Source Code}
  \begin{lstlisting}
    \ProvidesPackage{keisennote}[2025/10/05, v1.0.4a]

    \RequirePackage[dvipsnames, svgnames, x11names]{xcolor}
    \RequirePackage{zref, zref-savepos, fp}
    \RequirePackage{tikz}

    \RequirePackage{kvoptions} 

    \SetupKeyvalOptions{%
      family=kn,%
      prefix=kn@%
    }

    \newdimen\noteLineWidth
    \noteLineWidth=.5truept

    \newdimen\dotsRadius
    \dotsRadius=.8truept

    \newdimen\noteLineDistance
    \noteLineDistance=6truemm

    \newdimen\VoD@mag
    \VoD@mag=.5pt

    %%%パッケージオプションの宣言
    \DeclareStringOption[.5truept]{linewidth}% 線の太さ
    \DeclareStringOption[.8truept]{radius}% ドットの大きさ
    \DeclareStringOption[6truemm]{distance}% ドットの間隔
    \DeclareStringOption[.5pt]{triangle}% 三角形の大きさ

    \ProcessKeyvalOptions* % オプション適用

    %%%オプションの反映
    \setlength{\noteLineWidth}{\kn@linewidth}
    \setlength{\dotsRadius}{\kn@radius}
    \setlength{\noteLineDistance}{\kn@distance}
    \setlength{\VoD@mag}{\kn@triangle}


    %%%途中でパラメータ変更ができるように
    \NewDocumentCommand{\SetNoteLineWidth}{O{.5truept}}{%
      \setlength{\noteLineWidth}{#1}
    }
    \NewDocumentCommand{\SetNoteDotRadius}{O{.8truept}}{%
      \setlength{\dotsRadius}{#1}
    }
    \NewDocumentCommand{\SetNoteLineDistance}{O{6truemm}}{%
      \setlength{\noteLineDistance}{#1}
    }
    \NewDocumentCommand{\SetNoteTriangleSize}{O{.5pt}}{%
      \setlength{\VoD@mag}{#1}
    }

    %%%必要な内部レジスタの用意
    \newdimen\VDNT@currentXPos
    \newdimen\VDNT@currentYPos
    \newdimen\VDNT@Xinterval
    \newdimen\VDNT@Yinterval
    \newdimen\VDNT@notegoal

    %%% \notefillで用いる座標管理用カウンタの準備
    \def\VDNT@pkgname{vodnote}
    \global\newcount\VDNT@uniqe


    %%%%%%%%%%%%%%%%%%%%%%%%%%%%%%%%%%%%%%%%%%%%%%%%%%%%%%%%%%%%%%%%%%%%%%%%%%%%%%%%%%%%%%%%%
    %%% \notefill の定義
    \NewDocumentCommand{\notefill}{ O{white!70!black} }{\par\bgroup
      \parindent\z@
      %%罫線間隔の算出
      \@tempcnta\linewidth
      \@tempcntb\noteLineDistance
      \FPeval\VDNT@dotsNum{round(round(((\the)\@tempcnta/(\the)\@tempcntb)/2:0)*2:0)}%
      \VDNT@Xinterval\dimexpr(\linewidth)/\VDNT@dotsNum\relax
      \VDNT@Yinterval\VDNT@Xinterval
      %%上端の座標取得
      \zsaveposy{\VDNT@pkgname.\the\VDNT@uniqe.TopPos}%
      %%下端の座標取得
      \leavevmode\vfill\leavevmode
      \zsaveposy{\VDNT@pkgname.\the\VDNT@uniqe.BottomPos}%
      %%ノート罫線描画幅の決定
      \VDNT@notegoal=\dimexpr
        \zposy{\VDNT@pkgname.\the\VDNT@uniqe.TopPos}sp
        -\zposy{\VDNT@pkgname.\the\VDNT@uniqe.BottomPos}sp
      \relax
      %%ノート罫線描画
      \noindent\smash{%
        \begin{tikzpicture}[xscale=0.996]
          \VDNT@currentYPos\z@
          \fill[#1] (\VDNT@Xinterval*\VDNT@dotsNum/2,\VDNT@currentYPos+\VoD@mag*4pt) -- ++(\VoD@mag*3pt,-\VoD@mag*4pt) -- ++(-\VoD@mag*6pt,0) -- cycle;
          \@whiledim\VDNT@currentYPos<\VDNT@notegoal\do{
            \VDNT@currentXPos\z@
            \draw[#1,line width=\noteLineWidth] (0,\VDNT@currentYPos) -- (\linewidth,\VDNT@currentYPos);
            \foreach \k in{0,1,...,\VDNT@dotsNum}{%
              \VDNT@currentXPos=\dimexpr\VDNT@Xinterval*\k\relax
              \fill[#1] (\VDNT@currentXPos,\VDNT@currentYPos) circle [radius=\dotsRadius];
            }
            \advance\VDNT@currentYPos\VDNT@Yinterval\relax
          }
          \fill[#1] (\VDNT@Xinterval*\VDNT@dotsNum/2,\VDNT@currentYPos-\VDNT@Yinterval-\VoD@mag*4pt) -- ++(\VoD@mag*3pt,\VoD@mag*4pt) -- ++(-\VoD@mag*6pt,0) -- cycle;
        \end{tikzpicture}%
      }%
      \egroup
      %%座標管理用カウンタのインクリメント
      \global\advance\VDNT@uniqe\@ne
      \par
    }


    %%% \note の定義(2以上の整数を引数に)
    \NewDocumentCommand{\note}{ m O{white!70!black} }{\par\bgroup
      %%罫線間隔の算出
      \@tempcnta\linewidth
      \@tempcntb\noteLineDistance
      \FPeval\VDNT@dotsNum{round(round(((\the)\@tempcnta/(\the)\@tempcntb)/2:0)*2:0)}%
      \VDNT@Xinterval\dimexpr\linewidth/\VDNT@dotsNum\relax
      \VDNT@Yinterval\VDNT@Xinterval
      %%ノート罫線描画
      \noindent
        \begin{tikzpicture}[xscale=0.996]
          \VDNT@currentYPos\z@
          \fill[#2] (\VDNT@Xinterval*\VDNT@dotsNum/2,\VDNT@currentYPos+\VDNT@Yinterval+\VoD@mag*4pt) -- ++(\VoD@mag*3pt,-\VoD@mag*4pt) -- ++(-\VoD@mag*6pt,0) -- cycle; %上の三角形
          \foreach \i in{1,2,...,#1}{ 
            \VDNT@currentXPos\z@
            \global\VDNT@currentYPos=\dimexpr\VDNT@Yinterval*\i\relax
            \draw[#2,line width=\noteLineWidth] (0,\VDNT@currentYPos) -- (\linewidth,\VDNT@currentYPos);
            \foreach \k in{0,1,...,\VDNT@dotsNum}{
              \VDNT@currentXPos=\dimexpr\VDNT@Xinterval*\k\relax
              \fill[#2] (\VDNT@currentXPos,\VDNT@currentYPos) circle [radius=\dotsRadius];
            }
          }
          \fill[#2] (\VDNT@Xinterval*\VDNT@dotsNum/2,\VDNT@currentYPos-\VoD@mag*4pt) -- ++(\VoD@mag*3pt,\VoD@mag*4pt) -- ++(-\VoD@mag*6pt,0) -- cycle; %下の三角形
        \end{tikzpicture}%
      \egroup
      \par
    }


    %%%%%%%%%%%%%%%%%%%%%%%%%%%%%%%%%%%%%%%%%%%%%%%%%%%%%%%%%%%%%%%%%%%%%%%%%%%%%%%%%%%%%%%%%
    \NewDocumentCommand{\masumefill}{ O{white!70!black} }{\par\bgroup
      \parindent\z@
      %%罫線間隔の算出
      \@tempcnta\linewidth
      \@tempcntb\noteLineDistance
      \FPeval\VDNT@dotsNum{round(round(((\the)\@tempcnta/(\the)\@tempcntb)/2:0)*2:0)}%
      \VDNT@Xinterval\dimexpr(\linewidth)/\VDNT@dotsNum\relax
      \VDNT@Yinterval\VDNT@Xinterval
      %%上端の座標取得
      \zsaveposy{\VDNT@pkgname.\the\VDNT@uniqe.TopPos}%
      %%下端の座標取得
      \leavevmode\vfill\leavevmode
      \zsaveposy{\VDNT@pkgname.\the\VDNT@uniqe.BottomPos}%
      %%ノート罫線描画幅の決定
      \VDNT@notegoal=\dimexpr
        \zposy{\VDNT@pkgname.\the\VDNT@uniqe.TopPos}sp
        -\zposy{\VDNT@pkgname.\the\VDNT@uniqe.BottomPos}sp
      \relax
      %%ノート罫線描画
      \noindent\smash{%
        \begin{tikzpicture}[xscale=0.996]
          \VDNT@currentYPos\z@
          \fill[#1] (\VDNT@Xinterval*\VDNT@dotsNum/2,\VDNT@currentYPos+\VoD@mag*4pt) -- ++(\VoD@mag*3pt,-\VoD@mag*4pt) -- ++(-\VoD@mag*6pt,0) -- cycle;
          \@whiledim\VDNT@currentYPos<\VDNT@notegoal\do{
            \VDNT@currentXPos\z@
            \draw[#1,line width=\noteLineWidth] (0,\VDNT@currentYPos) -- (\linewidth,\VDNT@currentYPos);
            \foreach \k in{0,1,...,\VDNT@dotsNum}{%
              \VDNT@currentXPos=\dimexpr\VDNT@Xinterval*\k\relax
              \draw[#1,line width=\noteLineWidth]
              (\VDNT@currentXPos,0) -- (\VDNT@currentXPos,\VDNT@currentYPos);
              \fill[#1] (\VDNT@currentXPos,\VDNT@currentYPos) circle [radius=\dotsRadius];
            }
            \advance\VDNT@currentYPos\VDNT@Yinterval\relax
          }
          \fill[#1] (\VDNT@Xinterval*\VDNT@dotsNum/2,\VDNT@currentYPos-\VDNT@Yinterval-\VoD@mag*4pt) -- ++(\VoD@mag*3pt,\VoD@mag*4pt) -- ++(-\VoD@mag*6pt,0) -- cycle;
        \end{tikzpicture}%
      }%
      \egroup
      %%座標管理用カウンタのインクリメント
      \global\advance\VDNT@uniqe\@ne
      \par
    }


    \NewDocumentCommand{\masume}{ m O{white!70!black} }{\par\bgroup
      %%罫線間隔の算出
      \@tempcnta\linewidth
      \@tempcntb\noteLineDistance
      \FPeval\VDNT@dotsNum{round(round(((\the)\@tempcnta/(\the)\@tempcntb)/2:0)*2:0)}%
      \VDNT@Xinterval\dimexpr\linewidth/\VDNT@dotsNum\relax
      \VDNT@Yinterval\VDNT@Xinterval
      %%ノート罫線描画
      \noindent
        \begin{tikzpicture}[xscale=0.996]
          \VDNT@currentYPos\z@
          \fill[#2] (\VDNT@Xinterval*\VDNT@dotsNum/2,\VDNT@currentYPos+\VDNT@Yinterval+\VoD@mag*4pt) -- ++(\VoD@mag*3pt,-\VoD@mag*4pt) -- ++(-\VoD@mag*6pt,0) -- cycle; %上の三角形
          \foreach \i in{1,2,...,#1}{ 
            \VDNT@currentXPos\z@
            \global\VDNT@currentYPos=\dimexpr\VDNT@Yinterval*\i\relax
            \draw[#2,line width=\noteLineWidth] (0,\VDNT@currentYPos) -- (\linewidth,\VDNT@currentYPos);
            \foreach \k in{0,1,...,\VDNT@dotsNum}{
              \VDNT@currentXPos=\dimexpr\VDNT@Xinterval*\k\relax
              \draw[#2,line width=\noteLineWidth] (\VDNT@currentXPos,\VDNT@Yinterval) -- (\VDNT@currentXPos,\VDNT@Yinterval*#1);
              \fill[#2] (\VDNT@currentXPos,\VDNT@currentYPos) circle [radius=\dotsRadius];
            }
          }
          \fill[#2] (\VDNT@Xinterval*\VDNT@dotsNum/2,\VDNT@currentYPos-\VoD@mag*4pt) -- ++(\VoD@mag*3pt,\VoD@mag*4pt) -- ++(-\VoD@mag*6pt,0) -- cycle; %下の三角形
        \end{tikzpicture}%
      \egroup
      \par
    }

    \endinput
  \end{lstlisting}
\end{document}
